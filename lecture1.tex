\documentclass[a4paper,12pt]{article} 
\usepackage[T2A]{fontenc}			
\usepackage[utf8]{inputenc}			
\usepackage[english,russian]{babel}	
\usepackage{amsmath,amsfonts,amssymb,amsthm,mathtools} 
\usepackage{wasysym}
\usepackage{amsmath}

\author{конспект от TheLostDesu}
\title{Системы линейных алгебраических уравнений и матрицы}
\date{\today}


\begin{document}
\maketitle
\section{Определение}
Матрицей размера $m x n$ называется упорядоченная прямоугольная таблица содержащяя m строк  и n столбцов.\\

A = $\begin{pmatrix}
	a_11 & a_12 & ... a_1n \\
	... & ... & ... & ...\\
	(am1 & am2 & ... & amn)
\end{pmatrix}$

$a_{ij}$ - элемент матрицы\\ 
$i$ - номер строки\\
$j$ - номер столбца\\
$m$ и $n$ называют размерами матрицы
$[A]ij = aij$\\

\section{Частные случаи матриц}
Квадратная матрица\hspace{20pt}  ($m = n$)\\
$m$-мерный столбец\hspace{20pt} $n = 1$\\
$n$-мерная строка\hspace{20pt} $m = 1$\\
Нулевая матрица\hspace{20pt} все $a_ij$ = 0\\
Еденичная матрица \hspace{19pt} квадратная матрица,$\forall i = \overline{1, m}, j = \overline{1, n}$ $a_{ij} = \delta^i_j$ \footnote{Символ кронекера - $\delta^i_j$. Равен $1$, когда $i = j$, иначе равен 0}\\
В единичной матрице на диагонали стоят единицы, на остальных местах - нули.

\section{Операции с матрицами}
Две матрицы $A$ и $B$ называются равными, если они одинакового размера и соответствующие элементы матриц равны.\\
$\forall i = \overline{1, m}, j = \overline{1, n}$ $a_{ij} = b_{ij}$

Матрица $C$ называется суммой матриц $A$ и $B$, если матрицы $A$, $B$ и $С$ одинаковых размеров, и $c_{ij} = a_{ij} + b_{ij} \forall i = \overline{i, m}, j = \overline{1, n}$. $C = A+B$. Сложение матриц - коммуникативно, так как сложение элементов коммуникативно. 
Сложение матриц ассоциативно, так как сложение элементов ассоциативно. Сложение матрицы с нулевой матрицой = самой матрице.

Матрица $C$ называется произведением числа $\lambda$ на матрицу $A$, если  матрицы $C$ и $A$ одинаковых размеров, и $c_{ij} = \lambda \cdot a_{ij}$. C = $\lambda \cdot A$.

Транспонированием матрицы называется операция, переводящяя все строки все строки в столбцы с сохранением порядка следования. $A^t$. Матрица типа $m\times n$ переходит в матрицу $n\times m$. Матрица называется симметрической, если $A = A^t$

Произведением матриц $A_{n\times p}$ и $B_{p\times k}$ называется матрица $C$ размера $n\times k$, где $\forall i = \overline{1, m}, j = \overline{1, n}$ $C_{ij} = \sum\limits_{q=1}^p a_{iq} \cdot b_{qj}$


\end{document}