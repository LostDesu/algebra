\documentclass[a4paper,12pt]{article} 
\usepackage[T2A]{fontenc}			
\usepackage[utf8]{inputenc}			
\usepackage[english,russian]{babel}	
\usepackage{amsmath,amsfonts,amssymb,amsthm,mathtools} 
\usepackage{wasysym}
\usepackage{amsmath}
\everymath{\displaystyle}

\author{конспект от TheLostDesu}
\title{Матрицы}
\date{\today}


\begin{document}
\maketitle
\section{Алгоритм приведения матрицы к ступенчатому виду}
1. Найти первый ненулевой элемент в строке i - это <<ведущий элемент>>.\\
2. Теперь для k - ой строки берем число $-\frac{A_{kj}}{A_{ij}}$. И с этим коэфицентом\\ прибавляем i-ю строку к k-ой.\\
3. i-ю строку делим на значение ведущего элемента.(для улучшеного ступенчатого вида).\\
4. Выбираем новый элемент смещаясь в матрице на следующую строку.\\
5. Повторяем до того, как 4е действие сделать возможно.\\

Почему этот алгоритм всегда работает?\\
Так как матрица имеет конечные размеры, в частности конечное число столбцов, а за 1 шаг алгоритма в одном из столбцов на всех местах кроме i становятся нули(значит, что за шаг мы перемещаемся минимум на один столбец). Значит, что процесс закончится. Алгоритм называется <<Методом Гаусса>>
\section{Решение систем линейных уравнений методом Гаусса}
Пусть есть несколько уравнений(не обязательно столько же, сколько переменных) вида:
$a_1 * x_1 + a_2 * x_2... + a_n * x_n = b_1$. Назовем это координатной формой записи. 
Заметим, что система таких уравнений - произведение матрицы системы на матрицу с $x_1...x_n$.

Назовем расширеной матрицей системы матрицу вида 
\[
\begin{matrix}
a_11  & a_12 & ... & a_1n & b_1\\
a_21  & a_22 & ... & a_2n & b_2\\
a_31  & a_32 & ... & a_3n & b_3\\
...
\end{matrix}
\]

\section{Определитель матрицы}
Всякое расположение чисел от одного до n в любом порядке называют перестановкой.\\
Инверсия - случай, когда $\alpha_j > \alpha_i$, но, $i > j$
Знак перестановки = $(-1)^{кол-во инверсий в перестановке} sgn(\alpha)$\\
Транспозиция  - преобразование, когда в перестановке меняются местами два элемента, остальные остаются на своих местах. Любая транспозиция меняет четность перестановки.\\
Подстановка. 
$\sigma = \begin{matrix}
1 & ... & n\\
\sigma(1) & ... & \sigma(n)
\end {matrix}$
Т.е биекция чисел от одного до n в себя. Знаком подстановки называют знак перестановки в нижней строке.
Есть несколько вариантов записи: например, последовательно записать вершины в циклах вот так: (134)(2). \\
Если $\sigma$ - подстановка сама в себя, то мы называем ее тождественной. Обозначается за id. \\
На множесте подстановок можно ввести умножение: последовательное применение(композицию отображений).

Определитель(детерминант) квадратной матрицы.(det(A)) = $\sum_{\sigma \in S_n} sgn(\sigma) * a_{1\sigma(1)}*a_{2\sigma(2)}...*a_{n\sigma(n)}$. 
\end{document}