\documentclass[a4paper,12pt]{article} 
\usepackage[T2A]{fontenc}			
\usepackage[utf8]{inputenc}			
\usepackage[english,russian]{babel}	
\usepackage{amsmath,amsfonts,amssymb,amsthm,mathtools} 
\usepackage{wasysym}
\usepackage{amsmath}
\everymath{\displaystyle}

\author{конспект от TheLostDesu}
\title{Определители матриц}
\date{\today}


\begin{document}
\maketitle
\section{Свойства определителей}\
1. $det(A^T) = det(a)$\footnote{Это значит, что все свойства строк справедливы для столбцов, и наоборот.}\\
2. Определитель линеен по столбцам. То есть $det(A_1, A_2...\alpha * A_i...A_n) = \alpha * det(A_1, A_2...A_i...A_n)$\\
3. При перестановке столбцов определитель меняет знак\footnote{Говорят, что определитель является кососиметрической функцией столбцов.}\\
4. определитель равен нулю, когда: есть нулевая строка или две строки равны.\\
5. Определитель равен нулю, когда одна из строк равна линейной комбинацией остальных строк\footnote{Говорят, что j-тая строка является линейной комбинацией остальных, если каждый элемент ее является суммой элементов из остальных строк стоящих на той же позиции с какими-то коэфицентами.}\\
6. Определитель не изменится, если к строке прибавить линейную комбинацию остальных строк.
7. Определитель еденичной матрицы = 1
8. Любая ф-ия от столбцов матрицы удвлетворяющая св-вам 2,3 и 7 - определитель.
\end{document}