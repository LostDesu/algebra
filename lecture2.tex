\documentclass[a4paper,12pt]{article} 
\usepackage[T2A]{fontenc}			
\usepackage[utf8]{inputenc}			
\usepackage[english,russian]{babel}	
\usepackage{amsmath,amsfonts,amssymb,amsthm,mathtools} 
\usepackage{wasysym}
\usepackage{amsmath}

\author{конспект от TheLostDesu}
\title{Матрицы}
\date{\today}


\begin{document}
\maketitle
\section{Свойства матриц}
1) Сложение матриц коммуникативно
2) Сложение матриц ассоциативно.
3) Существует нейтральный элемент относительно сложения матриц. Это - нулевая матрица
4) Для любой матрицы A найдется матрица B, что A + B = 0
5) $(\alpha * \lambda) * A  = A * \alpha * \lambda$ 
7) $\alpha $ * ($A + B$) = $\alpha A +\alpha B$\\
8) 1 * $A$ = $A$

Свойства транспонирования\\
1) $(A^T)^T = A$\\
2) $(A+B)^T = A^T+B^T$\\
3) $(\lambda * A)^T = \lambda * A ^ T$\\
4) $(A * B) ^ T = B^T * A^T$

Свойства умножения.\\
Все матрицы в свойствах возможно умножить(У них верные размерности для умножения, и элементы возможно умножить).\\
1) A, B, C (A*B)*C = A*(B*C)\\
Т.Е две матрицы равны.\\
2) (A+B) * C = A * C + B * C и D * (A+B) = D * A + D * B\\
3) Существует нейтральный элемент по умножению для квадратной матрицы. Это еденичная матрица\footnote{матрица, у которой на главной диагонали стоят еденицы, а все остальные элементы - нули}.\\
4) Если умножить квадратную матрицу на нулевую, то ответ будет нулевой матрицей.\\
5) Пусть есть матрицы A и B типов n * m и m * k. Тогда $(A * B)^T$ = $B^T * A ^T$\\
Рассмотрим произвольный элемент матрицы-результата. Он равен $\sum^n_{r=1} \left[ A^T \right] _{jr} \left[ B^T \right] _ri$. но это равно и $\left[ B^T * A^T \right]$ по определению умножения. Значит, что свойство 5 работает.\\
7) Умножение матриц не коммутативно. Т.Е A * B != B * A. Даже если оба произведения определены. Пример: \\
\\
1 1 * 0 0 = 1 1\\  
0 0 * 1 1 = 0 0\\
\\
0 0 * 1 1 = 0 0\\
1 1 * 0 0 = 1 1\\

Элементарные преобразования:\\
1) Умножение i-той строки матрицы на число не равное нулю.\\
2) Перестановка двух строк в матрице\\
3) Прибавление к i-той строке k-ой строки с коэфицентом лямбда.\\

Каждое элементарное преобразование можно обратить.

Каждое элементарное преобразование матрицы можно трактовать, как умножение матриы на матрицу специального вида.\\
Эта матрица получается если то же самое преобразование выполнить над еденичной матрицей.\\
Матрица ступенчатого вида - номера первых ненулевых элементов всех строк (ведущие элементы) возрастают, а нулевые строки стоят внизу матрицы.\\
Матрица канонического вида - матрица ступенчатого вида, в которой все ведущие элементы = 1 и в любом столбце с ведущим элементом стоят только нули.\\

Любую конечную матрицу можно привести элементарными преобразованиями к ступенчатому виду(и к каноническому).
\end{document}